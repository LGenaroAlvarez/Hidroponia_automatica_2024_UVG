\noindent \textbf{Prefacio} \\
\\
\textbf{Lista de figuras} \\
\\
\textbf{Lista de cuadros} \\
\\
\textbf{Resumen} \\
\\
\textbf{Abstract}

\begin{enumerate}
	\item \textbf{Introducción}
	\item \textbf{Antecedentes}
	\begin{enumerate}
		\item La hidroponía en Guatemala
		\item Tendencias actuales en la producción sostenible de hortalizas
		\item Uso de \textit{Fuzzy Logic} para el control de sistemas hidropónicos
		\item Monitoreo y control de cultivos hidropónicos utilizando tecnología IoT
		\item La digitación de la agricultura
	\end{enumerate}
	\item \textbf{Justificación}
	\item \textbf{Objetivos}
	\begin{enumerate}
		\item Objetivo general
		\item Objetivos específicos
	\end{enumerate}
	\item \textbf{Alcance}
	\item \textbf{Marco Teórico}
	\begin{enumerate}
		\item Cultivos hidropónicos
		\item Nutrient Film Technique (NFT)
		\item Solución de nutrientes
		\item Características de la solución nutritiva del cilantro
		\item Monitoreo de parámetros para el crecimiento de plantas
		\item Placa de desarrollo ESP32
		\item Sistemas de control y conectividad con la nube
		\begin{itemize}
			\item El internet de las cosas (IoT)
			\item Protocolo de comunicación HTTP
			\item Comunicación WiFi con \textit{Dweet.io}
			\item Controladores de \textit{Fuzzy Logic}
		\end{itemize}
	\end{enumerate}
	\item \textbf{Análisis del crecimiento y rendimiento del cilantro en agricultura tradicional}
	\item \textbf{Elaboración de sistema hidropónico}
	\begin{enumerate}
		\item Diseño y construcción de la estructura
		\item Programación de sistemas embedidos
		\item Recolección de datos y calibración de sensores
		\item Diseño y programación de los sistemas de control
		\item Pruebas de crecimiento en el sistema hidropónico
	\end{enumerate}
	\item \textbf{Programación de aplicación móvil}
	\begin{enumerate}
		\item Conectividad a \textit{Dweet.io} mediante WiFi
		\item Pantalla de monitoreo
		\item Pantalla de control de parámetros
	\end{enumerate}
	\item \textbf{Sistema Final}
	\item \textbf{Conclusiones}
	\item \textbf{Recomendaciones}
	\item \textbf{Bibliografía}
\end{enumerate}

