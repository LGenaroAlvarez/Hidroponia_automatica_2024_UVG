\subsection*{Definición de línea base de crecimiento del cilantro}
Es importante cultivar una muestra de cilantro empleando una técnica de crecimiento tradicional con sustrato nutritivo con tal de comparar el rendimiento del cilantro hidropónico cultivado en el sistema propuesto. El cultivo de control o de línea base se realizará utilizando un sustrato definido en un período de tiempo preestablecido. Al finalizar el período de crecimiento del cultivo, se realizará una cosecha del cilantro durante la cual se realizarán mediciones de sus características físicas. Entre estas se incluirá el peso húmedo de la planta (esencialmente el peso de la planta incluyendo raíces y tallo principal), la cantidad de hojas presentes, y su altura total desde la base del tallo principal. Estos parámetros se registrarán para realizar una comparación cuantitativa y cualitativa del cilantro hidropónico versus el cilantro cultivado en sustrato nutritivo.

Tanto el cultivo de cilantro en sustrato nutritivo como en el sistema hidropónico se iniciarán desde la semilla en un sustrato de germinación. Se utilizarán semillas de la tienda Hidroponía Guatemala, las cuales se germinarán en una toalla de papel húmeda resguardada en una bolsa \textit{zip-lock}. En el caso del sustrato nutritivo, se iniciarán 10 semillas, de las cuales se seleccionarán 5 brotes para ser cultivados en un área designada. Luego de un período de una semana, se revisarán las semillas para seleccionar las que hayan germinado y trasplantarlas a un sustrato de incubación. Este consistirá en una mezcla de compost orgánico, perlita, y fibra de coco fina. Adicionalmente, se agregará canela en polvo a la mezcla para reducir la probabilidad de crecimiento de hongos o moho en las semillas o raíces. Una vez que los retoños hayan alcanzado una altura aproximada de cinco centímetros, se  revisarán las raíces para determinar si estas se han desarrollado lo suficiente para ser trasplantadas. En caso de que el crecimiento sea satisfactorio, se trasplantarán a un área de cultivo con un espacio entre plantas de 150 mm. 

El sustrato de cultivo consistirá de una mezcla de arena de río, tierra negra, abono orgánico y fibra de coco. Al igual que en el sustrato de crecimiento, se agregará canela en polvo para reducir la probabilidad del desarrollo de hongos o moho en las raíces. Se realizará un monitoreo diario de las plantas con un riego del sustrato cada segundo día de ser necesario. Se definirá un período de crecimiento de 55 días, durante el cual se tomarán medidas preventivas para evitar plagas como el uso de pesticidas orgánicos. Al finalizar dicho período, se realizará la cosecha del cilantro, retirando el mismo de raíz. Antes de realizar el pesado se lavarán cuidadosamente las plantas, particularmente las raíces, para eliminar restos de sustrato o piedras.

Se registrará el peso del cilantro cosechado incluyendo las raíces con tal de obtener el peso húmedo de la cosecha. Este peso se obtendrá mediante escalas de precisión para considerar leves variaciones en el peso entre plantas. Adicionalmente, se registrará la longitud de las raíces, la altura total del cilantro desde la base del tallo principal, y se realizará un recuento de las hojas para determinar su densidad. Este proceso se realizará para cada una de las cinco plantas sembradas y se almacenarán los datos para ser comparados luego de obtener los resultados de las pruebas con el sistema hidropónico automático.

\subsection*{Construcción del sistema hidropónico}
El sistema hidropónico consistirá en tres etapas o módulos los cuales serán integrados para obtener las condiciones de crecimiento deseadas y el control de parámetros. Estos módulos cumplen diferentes funcionalidades, desde proporcionar una estructura estable al sistema hasta realizar las mediciones y el control de parámetros. La construcción del sistema hidropónico requerirá de la integración de estos módulos para su funcionamiento adecuado.

\subsubsection{Estructura física principal}
Esta estructura consistirá en una estantería de metal con repisas fijadas a una distancia aproximada de 400 mm entre sí. Estas repisas cumplirán la función de proporcionar una base estable para colocar los canales de crecimiento y los sensores del sistema. Adicionalmente, las repisas proveerán una superficie en la cual se colocaran las luces de crecimiento. Con tal de facilitar el proceso de construcción, se utilizará una repisa prefabricada de metal y madera, la cual cuenta con alturas de repisas ajustables. El metal con protección ambiental reducirá los posibles daños por corrosión, y la madera provee una plataforma flexible sobre la cual se pueden montar los diferentes elementos del sistema. En la Figura \ref{figmesh:1} se puede apreciar la estructura principal.

Si bien basta con la estantería para soportar el sistema hidropónico vertical, esta deberá ser cubierta con tal de permitir un control más cercano de los niveles de luz en el sistema así como de la temperatura ambiente y humedad. Se utilizará un nailon negro en pliegos para recubrir la estructura, utilizando acoples para tensar el material y lograr una cubertura adecuada. Adicionalmente, se agregarán dos agujeros para incluir ventiladores en la estructura, los cuales proveerán una circulación de aire y serán empleados para controlar la temperatura y humedad interna.

\subsubsection{Distribución de solución nutritiva}
El sistema de distribución de nutrientes consistirá en una serie de tuberías de PVC las cuales distribuirán la solución nutritiva desde un depósito en la base del sistema hacia todas las repisas de crecimiento. Se utilizarán tuberías de PCV con un diámetro de 3 pulgadas colocados a 150 mm entre centros. Estas almacenarán las plantas, por lo que tendrán un agujero de 2 pulgadas de diámetro en el cual se colocarán las canastas de crecimiento. Adicionalmente, estas estarán inclinadas a un ángulo adecuado para facilitar la caída del agua de una repisa hacia otra. Esta inclinación se logrará mediante soportes impresos con filamento de impresión 3D para asegurar que las tuberías se mantengan fijas en la inclinación deseada. 

Los canales de crecimiento serán alimentados con solución nutritiva mediante un sistema de tuberías de PVC de $1/4$ de pulgada. Se utilizará una tubería principal a la cual se conectará una bomba de agua sumergible, y llevará la solución nutritiva hacia la repisa superior del sistema. Luego, se realizarán conexiones entre repisas utilizando tuberías del mismo diámetro para asegurar un flujo constante. Finalmente, la solución regresará al depósito inicial, el cual consistirá en una caja plástica. En esta se almacenará la solución nutritiva junto con los diferentes sensores y actuadores para controlar los parámetros de la solución nutritiva. Es importante mencionar que se permitirá una caída de por lo menos 150 mm para la solución que regresa al depósito, esto con tal de fomentar la agitación y oxigenación de la solución.

\subsubsection{Estructura de monitoreo y control}
Dentro del sistema hidropónico se estarán realizando mediciones de nivel de pH del agua, conductividad eléctrica de la solución, su temperatura, y la humedad y temperatura ambiente. La información de estos sensores será recolectada por un microcontrolador \textit{ESP WROOM32} el cual contendrá las funciones necesarias para la lógica de control y el accionamiento de los actuadores. Adicionalmente, permitirá la conectividad del sistema con la red inalámbrica para el monitoreo externo de parámetros desde una aplicación móvil. Estos sensores y el microcontrolador requerirán de un espacio en donde se puedan resguardar del agua y la humedad inherentes del sistema. Si bien, los sensores que se encontrarán en contacto directo con el agua están diseñados para ello, los demás componentes como la fuente de poder las placas de control no están diseñadas para resistir altos niveles de humedad. Por esta razón, se diseñará un contenedor dedicado y hermético para resguardar los componentes eléctricos de importancia.

El contenedor deberá ser capaz de almacenar una fuente de alimentación conmutada, la cual será utilizada para suministrar el voltaje requerido por los componentes del sistema. Adicionalmente, deberá admitir una conexión eléctrica general de 120VAC, la cual será el suministro de energía eléctrica para el sistema. Con tal de facilitar el diseño y la construcción del sistema, se evaluará el uso de un contenedor plástico prefabricado el cual cuente con un cierre hermético o sea fácil de sellar. Se utilizará silicona resistente al agua para rellenar cualquier abertura requerida para cables, y empaques en donde sea necesario. Adicionalmente, contará con un toma corrientes estándar para el suministro de energía eléctrica al sistema. Es importante mencionar que se deberán realizar cálculos de voltaje para asegurar que los cables utilizados en las etapas de potencia sean del calibre adecuado según las cargas esperadas. Finalmente, se diseñarán enganches requeridos para fijar los diferentes sensores en las ubicaciones deseadas alrededor del sistema hidropónico, los cuales serán impresos en 3D para reducir costos y tiempos de producción.

\subsection*{Programación del sistema embedido}
Tanto el control de los actuadores como la lectura de los sensores requerirá del uso de un microcontrolador. Considerando que se estará empleando el \textit{ESP WROOM32}, este se programará utilizando la plataforma de desarrollo de \textit{Arduino}. La programación del sistema embedido se dividirá en múltiples fases las cuales abarcan la programación de la comunicación inalámbrica para el envío y la recepción de datos de la nube, la lectura de sensores, los cálculos de control requeridos para estabilizar niveles deseados, y la programación de los actuadores. Cada una de estas etapas contará con pruebas a realizar, las cuales incluyen pruebas de conectividad y del funcionamiento correcto de sensores y actuadores.

\subsubsection{Conectividad a \textit{Dweet.io} mediante HTTP}
La ventaja de utilizar el sistema de \textit{Dweet.io} es que permite el uso de solicitudes mediante el protocolo HTTP. Esto aumenta la velocidad de la transmisión de datos al reducir la necesidad de establecer la conexión a un servidor predeterminado. Por otro lado, esto permite el desarrollo de dos funciones de comunicación dedicadas a solicitar y publicar información al servidor de \textit{Dweet.io}. Es importante establecer que una transferencia adecuada de datos se definirá como una transferencia sin pérdida de información, y que mantenga el orden estructural de los datos para facilitar su recuperación. 

La función utilizada para solicitar información deberá realizar diferentes operaciones al ser llamada. Primero, tendrá que generar la solicitud utilizando un paquete de texto predeterminado que incluya los componentes necesarios para solicitar los datos más actualizados almacenados en el servidor. La respuesta enviada por el servidor contará con una serie de datos los cuales deberán ser filtrados y organizados para obtener la información requerida. Por esta razón, la secunda operación que deberá realizar la función será separar el encabezado de la respuesta, y extraer los valores requeridos del paquete de texto. Finalmente, una vez se tengan los datos en el formato correcto, se deberán retornar estos para que puedan ser utilizados por el microprocesador en las operaciones de control.

De manera similar, la función utilizada para publicar información deberá realizar diferentes operaciones al ser llamada para asegurar que los datos sean transferidos adecuadamente. Primero, los valores ingresados a la función deberán ser organizados en un paquete de texto con el formato y la estructura establecida para almacenamiento en el servidor. Una vez formateados los datos, se realizará la publicación al servidor de \textit{Dweet.io} asegurando que se encuentre activa la conexión de internet necesaria. Luego, se deberán leer los encabezados enviados por el servidor como respuesta para asegurar que el paquete haya sido recibido correctamente, regresando un mensaje de éxito en el caso de que se haya ejecutado correctamente la transmisión. En caso de que la publicación de datos falle, se establecerá un tiempo de espera antes de realizar una segunda prueba. Si luego de el segundo intento se encuentra con un error, se devolverá un mensaje de error indicando que la información del servidor de \textit{Dweet.io} no pudo ser actualizada.

\subsubsection{Lectura y calibración de sensores}
El monitoreo de parámetros del sistema se realizará mediante una serie de sensores analógicos los cuales deberán ser integrados con el microcontrolador mediante puertos ADC (Convertidor Analógico a Digital). Adicionalmente, se cuenta con cinco sensores los cuales utilizan un protocolo conocido como \textit{onewire}, el cual utiliza un único cable para transmitir información del sensor a un microcontrolador. Una vez conectados todos los sensores, se podrán realizar pruebas de funcionamiento así como calibraciones cuando estas sean necesarias. En el caso del los sensores con comunicación \textit{onewire}, estos no requerirán de una calibración detallada. Por otro lado, los sensores analógicos como el sensor de pH y electro conductividad sí requieren de un proceso de calibración utilizando una solución especializada.

El proceso de calibración para los sensores analógicos requerirá del desarrollo de funciones específicas que sean capaces de almacenar los valores analógicos recibidos, y realizar la regresión lineal necesaria. Para esto se realizará una comparación entre el valor analógico recibido para una concentración dada de solución y su valor teórico de pH o electro conductividad según sea el caso. Luego, se almacenarán los resultados de las pruebas con las soluciones de control, y se realizará una medición con la solución del sistema. Este procedimiento deberá ser repetido de manera periódica según el sensor, por lo que se deberá incluir un sistema de notificación al usuario para re-calibrar los sensores luego de el tiempo definido de uso.

\subsubsection{Control de actuadores}
El control de la concentración de solución nutritiva y pH se realizará mediante el accionamiento de válvulas reguladoras de flujo. Ahora bien, con tal de reducir el costo de producción, estas válvulas se controlarán mediante servo motores, los cuales deberán ser acoplados las válvulas y configurados para lograr el rango de giro apropiado. Adicionalmente, se utilizarán servo motores para introducir el sensor de electro conductividad en la solución cada vez que se desee realizar una medición. Esto se debe a que es uno de los sensores más susceptibles al agua, por lo que es importante que se mantenga fuera de esta mientras que no se estén realizando mediciones.

Los servo motores se controlarán utilizando el \textit{ESP WROOM 32}, al igual que los sensores, por lo que será necesario configurar los puertos PWM (\textit{Pulse Width Modulated}). Estos permitirán el control preciso del ángulo de giro de cada uno de los servo motores, asegurando así que se definan posiciones angulares para cada solución de manera individual. La integración de estos sistemas junto con el algoritmo de control permitirá variar los parámetros del agua con nutrientes que será distribuido hacia las plantas. 

Si bien los servo motores juegan un papel crucial en el control de la concentración de nutrientes y los niveles de pH de la solución, no son los únicos actuadores presentes en el sistema. Con tal de variar la temperatura ambiente en el entorno de crecimiento, así como los niveles de humedad en el aire, se agregarán ventiladores los cuales permitirán un flujo de aire constante alrededor de las plantas. Estos ventiladores se controlarán utilizando relés los cuales deberán ser accionados por el microcontrolador seleccionado. Debido a esto, únicamente se contará con dos estados para los ventiladores, encendido y apagado, por lo que se deberá analizar el impacto del flujo máximo de aire entregado por los mismos en un período dado de tiempo. En base a este impacto, se deberán desarrollar funciones para accionar los ventiladores a manera de mantener un ambiente constante en los niveles deseados de humedad y temperatura ambiental. 

Finalmente, se contará con varias series de luces led controlables de la marca \textit{Adafruit}. Estas tiras de luces se utilizarán para generar ciclos de crecimiento y regular el consumo de nutrientes de las plantas. El microcontrolador deberá regular la luminosidad de las tiras, así como el color individual de cada una para asegurar que la frecuencia de luz entregada sea la óptima para el crecimiento del cultivo.

\subsection*{Diseño y evaluación del sistema de control}
El sistema hidropónico automático cuenta con una gran cantidad de variables que deben ser monitoreadas cuidadosamente. Por esta razón, se optó por considerar a cada uno de los sub-sistemas individuales para desarrollar e implementar los sistemas de control. En base a esto se definen tres sub-sistemas a controlar, los cuales regularán los niveles de nutrientes en el agua, el nivel de pH y la temperatura junto con la humedad ambiental respectivamente.

El proceso de control para cada uno de los parámetros consistirá en la definición de un período de muestreo general. Este se utilizará para determinar cambios en los parámetros a lo largo del día con una frecuencia relativamente baja. Luego, en el caso de que uno de los parámetros se encuentre fuera del rango aceptable definido, se aumentará el período de muestreo para determinar con mayor exactitud la el efecto de las variables de control sobre el parámetro a variar. Una vez logre estabilizar el sistema al rango preestablecido, se regresará al período de muestreo general. Esto se realizará con tal de reducir el consumo eléctrico del sistema, y alargar la vida útil de los sensores con mayor grado de degradación en agua. 

Una vez definido el período de muestreo de los sensores, se desarrollarán las funciones de control para cada uno de los sub-sistemas definidos. Debido a la complejidad matemática del modelado de cada sub-sistema para el ajuste de parámetros, se utilizarán controladores \textit{Fuzzy Logic}. Como primer paso, se deberán definir los rangos aceptables para cada uno de los parámetros a monitorear. Estos rangos permiten establecer las reglas de control y las relaciones entre sensores y actuadores. 

Finalmente, se realizará una evaluación del sistema de control realizando pruebas de regulación de parámetros. Para esto se iniciará un cronómetro en el momento en que los sensores realicen la lectura del parámetro a evaluar con tal de determinar el tiempo de estabilización del controlador. Adicionalmente, se evaluará la cercanía del valor de estado estable a los rangos deseados con tal de determinar si este se encuentra próximo a los límites, tanto inferior como superior.

\subsection*{Pruebas de rendimiento del sistema hidropónico}
Las pruebas de rendimiento del sistema hidropónico se realizarán una vez todas las etapas anteriores hayan sido completadas. Durante este período se buscará determinar la calidad producida de cilantro, y comparar esta con los resultados obtenidos de las pruebas realizadas con sustrato tradicional. Se empleara el siguiente procedimiento para realizar las pruebas controladas del sistema hidropónico completado.

El sistema hidropónico será cargado con solución nutritiva, reguladores de pH y agua. Con estos materiales se realizará un ciclo de cebado, tanto para asegurar los niveles de presión necesarios en las tuberías para el flujo de solución, como para estabilizar los parámetros iniciales en el sistema. Es importante considerar que los niveles de pH del agua utilizada dependerán de la ubicación y el tipo de tratamiento que reciba el agua disponible. Por esta razón, se prestará atención especial a nivelar estos valores antes de introducir las mezclas de nutrientes y las plantas para el proceso de crecimiento.

Al igual que las pruebas para la línea base del rendimiento del sustrato tradicional, se realizará un proceso de germinación. Para este se utilizará la misma mezcla de sustrato junto con la misma cantidad de semillas. Una vez germinadas, se esperará a que los retoños alcancen una altura de aproximadamente cinco centímetros. Una vez alcanzada dicha altura, se retirarán las plantas del sustrato y se limpiarán las raíces cuidadosamente utilizando agua para remover partículas de sustrato. Luego de esto, se colocarán en canastas de crecimiento hidropónico con perlita, la cual será utilizada como sustrato de anclaje para mantener fijas las plantas.

Los retoños de cilantro se mantendrán en el sistema hidropónico por 55 días, durante los cuales se realizarán observaciones periódicas de la salud del follaje así como la presencia de plagas. Junto a esto, se añadirá agua al sistema según sea necesario, considerando pérdidas por evaporación y absorción por parte de las plantas. Adicionalmente, se mantendrá un registro de los parámetros ambientales y de agua, los cuales serán almacenados en una base de datos local. Luego del período de crecimiento, se extraerán las plantas del sistema cuidadosamente, retirando primero las canastas, y luego separando las raíces de la perlita. Las plantas limpias y sin residuos de sustrato en sus raíces serán pesadas y medidas utilizando la misma metodología que en las pruebas de la línea base. Una vez se tengan estos resultados, se realizará una comparación entre los valores promedio encontrados para cada uno de los métodos de crecimiento. 

\subsection*{Desarrollo de aplicación móvil}
Si bien la aplicación móvil contará con la habilidad de controlar, de manera remota, los parámetros del sistema hidropónico como la concentración de nutrientes y el nivel de pH, esta será utilizada principalmente para monitorear el sistema. Para lograr esto se utilizará el programa de \textit{Android Studio} en el cual se programará la conectividad al internet, junto con los ciclos de solicitud de datos al servidor de \textit{Dweet.io} utilizado en el proyecto. Se configurarán como mínimo dos pantallas para la aplicación, una donde se contará con la información en tiempo real del sistema hidropónico, y otra en donde se podrán alterar los niveles deseados para cada uno de los parámetros disponibles. La finalidad de esta aplicación consistirá en permitir al usuario indagar en el estado de sus cultivos sin la necesidad de realizar una visita o mediciones manuales. El desarrollo de la aplicación se dividirá en las siguientes secciones.

\subsubsection{Conectividad al internet y prueba de comunicación con \textit{Dweet.io}}
Las habilidades de monitoreo y control de parámetros requerirán de una conexión estable al internet, por lo cual se utilizarán los sistemas nativos de \textit{Android Studio} para detectar si el dispositivo se encuentra conectado a una red de internet activa. Luego de esto, se activará la conexión de la aplicación a la red de internet, y se realizará una prueba de conectividad. Esta consistirá en la solicitud de la información almacenada en el servidor de \textit{Dweet.io}. En el caso de que no se cuente con una conexión activa al internet, o en caso de que no se logre recibir información adecuada del servidor de \textit{Dweet.io} se lanzará un mensaje de error en la aplicación.

\subsubsection{Pantalla de monitoreo}
Una vez se haya logrado la conexión al internet, y se estén recibiendo datos del servidor de Dweet.io, se habilitará la pantalla de monitoreo. En esta se encontrará la temperatura del agua en tanto el depósito como los canales de crecimiento, la concentración de nutrientes actual en $mS/cm^2$, el nivel de pH, la temperatura ambiental y la humedad en el sistema. Adicionalmente, se contará con un indicador del estado de las luces de crecimiento, mostrando si estas se encuentran encendidas o apagadas, así como la intensidad a la que se encuentran configuradas. Mientras que se encuentre activa esta pantalla, se realizará una solicitud de los parámetros mencionados con una frecuencia similar a la utilizada por el sistema para el monitoreo de parámetros.

\subsubsection{Pantalla de control de rangos aceptables}
Diferentes cultivos requieren de un control diferente de parámetros como nivel de pH en la solución o temperatura de la misma. Así mismo, es posible que a lo largo del período de crecimiento, sea necesario realizar modificaciones a los rangos en los cuales se desea mantener cada uno de los parámetros, en base a la respuesta de las plantas a su entorno. La segunda pantalla contará con un espacio en donde se podrán ingresar nuevos valores máximos y mínimos para cada uno de los parámetros. Estos rangos serán enviados a un servidor diferente de \textit{Dweet.io} y serán leídos por el microcontrolador para actualizar el sistema de control.


