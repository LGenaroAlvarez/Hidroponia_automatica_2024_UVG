La hidroponía ofrece grandes beneficios entre ellos, el ahorro significativo de agua, espacio de crecimiento y un control más robusto de las características de crecimiento del cultivo deseado. Ahora bien, estos sistemas requiere de cuidados específicos y diferentes a los de la agricultura tradicional, como el monitoreo de acidez o alcalinidad de la solución o su concentración de oxígeno. Como lo han demostrado varios estudios en el extranjero, contar con un sistema de regulación de parámetros autónomo permite obtener mejores resultados con sistemas hidropónicos. 

Este trabajo de graduación busca establecer una plataforma que pueda ser utilizada para analizar la viabilidad de sistemas hidropónicos automáticos para reforzar la producción de hortalizas locales, e incentivar su cultivo en contextos urbanos. Se desarrollará un sistema capaz de medir y controlar las diversas variables involucradas en el crecimiento de plantas con metodologías hidropónicas. Esta tecnología será validada mediante un prototipo del sistema a pequeña escala, diseñado con la intención de que sea adoptado en entornos urbanos para mejorar la calidad de las hortalizas consumidas. Mientras que el enfoque principal del trabajo recaerá en diseñar un sistema de bajo consumo eléctrico con costos reducidos, se buscará también optimizar la productividad del sistema. Adicionalmente, establecerá una línea base para futuras investigaciones que deseen aprovechar el control autónomo de parámetros para cultivos hidropónicos a grande y mediana escala. 

Se espera que el desarrollo de un sistema que cumpla con estas características, junto con los bajos requisitos de trabajo manual, aumente la atractividad de los sistemas hidropónicos en el mercado guatemalteco. Este tipo de sistemas serán esenciales para asegurar una producción estable de hortalizas en Guatemala, sin el riesgo pérdidas por cambios en condiciones climáticas. 