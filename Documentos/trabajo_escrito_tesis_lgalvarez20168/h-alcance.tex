

El presente proyecto de graduación buscó establecer las bases para el desarrollo y análisis de métodos de cultivo automático utilizando técnicas de hidroponía en la Universidad del Valle de Guatemala. Se realizó un estudio de las características de los sistemas hidropónicos, haciendo enfoque en la metodología de solución nutritiva re-circulante o NFT por sus siglas en inglés. Utilizando la información recopilada, se buscó desarrollar un prototipo que permitiera controlar diferentes parámetros relacionados al crecimiento de las plantas. Junto al desarrollo de un prototipo preliminar, se realizó una comparación de diferentes indicadores de crecimiento del cultivo seleccionado, tanto en el entorno hidropónico como en uno tradicional. Finalmente, se desarrolló una aplicación sencilla la cual permitiera monitorear los parámetros del sistema.