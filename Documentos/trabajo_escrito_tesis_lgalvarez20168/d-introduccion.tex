La hidroponía es un método de crecimiento de plantas sin la dependencia del suelo para suministrar los nutrientes requeridos para el desarrollo de los cultivos. Esto se logra al utilizar una solución de nutrientes disueltos en agua los cuales son suministrados constantemente a las raíces de las plantas. Existe una gran variedad de implementaciones de cultivos hidropónicos los cuales se caracterizan por su bajo consumo de agua, independencia del suelo y su eficiencia en el uso de espacio.

Si bien en Guatemala este método de cultivo se está empezando a implementar en diferentes regiones y sectores, como en el área de forraje para ganado, aún se encuentra en sus etapas iniciales. A pesar de ser un proceso prometedor para mejorar la seguridad alimenticia del país, estos sistemas requieren de un control y monitoreo de alta precisión en intervalos constantes. Por esta razón, los sistemas hidropónicos presentan retos al requerir mano de obra constante y con conocimientos elevados para realizar las mediciones y los cálculos necesarios para asegurar que los cultivos reciban los nutrientes esenciales.

Este proyecto de graduación busca desarrollar e implementar un sistema hidropónico automático, que sea capaz de monitorear y controlar una gama de parámetros, como el nivel de pH del agua y densidad de nutrientes, sin intervención humana constante. Para lograr esto se realizará una implementación a escala de un sistema para cultivos hidropónicos urbanos, que sea capaz de asegurar el crecimiento de una plata de cilantro en condiciones controladas. Adicionalmente, se implementará una interfaz gráfica demostrativa para que el sistema pueda ser monitoreado y controlado de manera remota desde un dispositivo móvil aprovechando el contexto de la red de las cosas.